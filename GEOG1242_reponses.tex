 \documentclass[12pt]{report}
   

 \renewcommand{\baselinestretch}{1.5}

\textheight22cm
\textwidth15cm
\hoffset-20mm
\voffset-25mm
\oddsidemargin2.5cm
\evensidemargin2.5cm
\usepackage[french]{babel}
\def\pa{\partial}    
\def\sun{\odot}
 
%%%%% \usepackage{amssymb}
%\input{option_keys}
%\usepackage{amsfonts}
\usepackage{amsmath}
\usepackage{amssymb}
\usepackage{epsfig}
\def\tan{\mathrm{tg}}    
\usepackage[latin1]{inputenc}
\usepackage[pdftex,bookmarks=true]{hyperref}
\title{Geographie math�matique GEOG1242 : solution des exercices \\\small{Notes provisoires ann�e 2007-2008}}
\author{Michel Crucifix}

\begin{document}

\maketitle
\underline{exercices}:  
\flushleft{
\begin{align*}
1' &= 4 s \\
1'' &= \frac{1}{15} s
\label{}
\end{align*}
}

\begin{equation*}
\cos(a-a)=\cos^2a + \sin^2a = 1
\end{equation*}

 \textbf{ tabuler les valeurs de $\varphi-\varphi'$, pour des latitudes de 0,30$^\circ$, 45$^\circ$, 60$^\circ$, 90$^\circ$}.

\begin{tabular}{l|l|l}
$\varphi\ (\ ^\circ)$ & $\varphi' (\ ^\circ)$ &  $\varphi - \varphi' (\ ^\circ) $  \\
\hline
    0.00&   0.00 &  0.00 \\
   15.00&  14.90 &  0.09 \\
   30.00&  29.83 &  0.16 \\
   45.00&  44.80 &  0.19 \\
   60.00&  59.83 &  0.16 \\
   75.00&  74.90 &  0.09 \\
   90.00&  90.00 &  0.00 
\end{tabular}

La diff�rence s'annule donc pour $0$ et $\pm90^\circ$ et est maximale pour $\pm45^\circ$.

\bigskip

\textbf{Que vaut l'orthodromie de deux points align�s sur un m�me m�ridien? }\\
On a $\delta\lambda=0$ et donc $\cos c = \cos (\varphi_A-\varphi_B)$, et 
 $c=| \varphi_A - \varphi_B|$.

\textbf{ Le cap � suivre par un marin suivant l'orthodromie vaut: $2\pi-A$ au d�part, et $B+\pi$ � l'arriv�e (attention : c'est bien $B+\pi$ et non $B$ comme l'a fait remarquer un �tudiant au cours). D�terminez $A$ et $B$ en utilisant les relations II.}

On peut utiliser 

\begin{equation*}
\frac{\sin\delta\lambda}{\sin c}=\frac{\sin B}{\cos\varphi_A}=\frac{\sin A}{\cos\varphi_B}
\end{equation*}

Donc, par exemple : 

\begin{equation*}
\sin B=\frac{\sin\delta\lambda \cos\varphi_A}{\sin c},
\end{equation*}

\begin{equation*}
B=\left\{
\begin{split}
\arcsin (\sin B) \\
\pi-\arcsin (\sin B)
\end{split}\right.
\end{equation*}

selon que la trajectoire est de l'Est vers l'Ouest, ou de l'Ouest vers l'Est. 

\bigskip

\textbf{
D�terminer la loxodromie et l'orthodromie entre Paris et New York, en utilisant les coordonn�es suivantes :}

Paris : $48^\circ$ 48'N  et $2^\circ20'$E  ; New York : $43^\circ03'$N $77^\circ�36'$W

Remarquons tout d'abord que $\varphi_A=48,8^\circ, \quad \varphi_B=43,05^\circ$ et $\delta\lambda=79,933^\circ$.
\bigskip

1) orthodromie : 

On trouve $\cos c =  0.59776$ et, en prenant un rayon terrestre de 6380\,km, on a une distance orthodromique de 5934 km.

2) loxodromie :

On applique la formule 
\begin{equation*}
\tan\gamma = \frac{\delta \lambda}{ \left[\ln\left (\tan \ ({\pi\over 4} + \frac{\varphi}{2})\right)\right]^A_B }
\end{equation*}

Pour trouver : $\gamma=1.467\mathrm{rad} = 84^\circ$ et la longueur de la loxodromie (en prenant $R=6380$km) vaut 6216 km (nous avons consid�r� l'orthodromie depuis New-York vers Paris pour obtenir un angle $\gamma$ dans le premier quadrant). 

\end{document}
